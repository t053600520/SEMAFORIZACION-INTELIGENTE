\documentclass[conference]{IEEEtran}
\IEEEoverridecommandlockouts
% The preceding line is only needed to identify funding in the first footnote. If that is unneeded, please comment it out.
\usepackage{cite}
\usepackage{amsmath,amssymb,amsfonts}
\usepackage{algorithmic}
\usepackage{graphicx}
\usepackage{textcomp}
\usepackage{xcolor}
\def\BibTeX{{\rm B\kern-.05em{\sc i\kern-.025em b}\kern-.08em
    T\kern-.1667em\lower.7ex\hbox{E}\kern-.125emX}}
\begin{document}

\title{Sistema de semaforización inteligente para el control de flujo vehicular mediante el Procesamiento Digital de Imágenes\\
{\footnotesize \textsuperscript{*}Note: Sub-titles are not captured in Xplore and
should not be used}
\thanks{Identify applicable funding agency here. If none, delete this.}
}
\author{\IEEEauthorblockN{1\textsuperscript{st} Cesar Camargo Lopez}
	\IEEEauthorblockA{\textit{Dpto. de Ingeniería Mecatrónica} \\
		\textit{Universidad Nacional de Trujillo}\\
		Trujillo, Perú \\
		txxxxxxxxxxx@unitru.edu.pe}
	\and
\IEEEauthorblockN{2\textsuperscript{nd} Eduar Cueva Blas}
\IEEEauthorblockA{\textit{Dpto. de Ingeniería Mecatrónica} \\
	\textit{Universidad Nacional de Trujillo}\\
	Trujillo, Perú \\
	t1023600221@unitru.edu.pe}
\and
\IEEEauthorblockN{3\textsuperscript{rd} Cesar Magán Armas}
\IEEEauthorblockA{\textit{Dpto. de Ingeniería Mecatrónica} \\
	\textit{Universidad Nacional de Trujillo}\\
	Trujillo, Perú \\
	txxxxxxxxxx@unitru.edu.pe}
}

\maketitle

\begin{abstract}
This document is a model and instructions for \LaTeX.
This and the IEEEtran.cls file define the components of your paper [title, text, heads, etc.]. *CRITICAL: Do Not Use Symbols, Special Characters, Footnotes, 
or Math in Paper Title or Abstract.
\end{abstract}

\begin{IEEEkeywords}
component, formatting, style, styling, insert
\end{IEEEkeywords}

\section{Introducción}
La congestión del tráfico es un grave problema que ha ido en aumento en los últimos años. En este contexto, la gestión eficaz del tráfico es un desafío para las grandes ciudades. Sin embargo, la implementación de proyectos de investigación dedicados a mejorar la señalización en zonas urbanas es una medida que puede evitar o en su gran mayoría reducir las pérdidas generadas por esta problemática \cite{Hoang2024}, \cite{Esra2024}, \cite{Victor2019}.

Los volúmenes de tráfico actuales no pueden ser manejados por métodos tradicionales de señales de tráfico. El congestionamiento afecta
directamente a los sectores de producción con pérdida de tiempo y recursos, la obstrucción de carreteras, acumulación de vehículos definen la duración de viajes que cada vez se hacen más largos. En los últimos años los sistemas inteligentes han adquirido una base sólida para la solución de problemas
cotidianos. El desarrollo de un sistema de semáforos inteligentes para el control de tráfico es posible mediante la implementación de estas tecnologías \cite{Tang2024}, \cite{Manzo2019}.

Algunos investigadores han desarrollado metodologías para la detección de
objetos en imágenes y video, mediante combinaciones de métodos y 
protocolos, a través de estas metodologías, se desarrolla una amplia variedad de proyectos innovadores para contrarrestas los problemas cotidianos, y en especial la problemática planteada. En el aporte de Monterrey y Sosa, desarrollaron un sistema inteligente mediante el uso de herramientas como Amazon Rekognition y Google Cloud obteniendo resultados favorables en simulación \cite{Segarra2022}, \cite{Monterrey2020}.

La implementación de redes neuronales y entornos virtuales de desarrollo, son tecnologías que cada vez se hacen más presentes en las soluciones a los grandes desafíos de la sociedad, tal es el aporte que no solo beneficia en el aprovechamiento de recursos y optimización de tiempo, además, permite una reducción de efectos negativos al medio ambiente, como lo es las emisiones de $CO_2$, aportes como el de Jacobo, desarrollaron un sistema inteligente para evaluar el tráfico en avenidas, obteniendo como resultado una reduccion en el tiempo de congestionamiento y reducción de emisiones contaminantes \cite{Alvarez2021}, \cite{Jacobo2015}.

Debido al contexto mencionado en esta investigación se tiene como objetivo mitigar los riesgos en la circulación, optimizar la fluidez del tránsito vehicular y disminuir el consumo energético de los vehículos que se desplazan por las vías mediante el uso de tecnologías de procesamiento de imágenes.

\section{Trabajo Relacionado}

Diversos artículos de investigación han utilizado SUMO para simular y optimizar la señalización vehicular. Hoang et al. \cite{Hoang2024} emplearon un algoritmo Max Weight, logrando una reducción del 32\% en el tiempo de espera y 22\% en emisiones contaminantes. Esra’a y Taqwa \cite{Esra2024} aplicaron un aprendizaje profundo por refuerzo, obteniendo resultados favorables, como la reducción del 56\% en emisiones de CO y más del 60\% en el consumo de combustible. Victor et al. \cite{Victor2019}, mediante la metodología y técnicas de simulación macroscópica y microscópica, reportaron reducciones en $CO_2$ (13,72\%) y en el número de vehículos (11,75\%) en horas pico. El estudio de Tang \cite{Tang2024} propuso un método utilizando control predictivo de modelos (MPC), GCNs y DRL generando una velocidad promedio mayor a un 6.4\% respecto al método tradicional de tiempo fijo. Estos trabajos demostraron la eficacia de SUMO en la optimización del tráfico y la reducción de impactos ambientales.

Manzo y Arzate \cite{Manzo2019} diseñaron un sistema de semáforos inteligentes basado en visión artificial y procesamiento de imágenes, utilizando Pycharm JetBrains y OpenCV, integraron sensores y reconocimiento de patrones en tiempo real. Este enfoque permitió la reducción de la congestión. Segarra y Torres\cite{Segarra2022} presentaron un algoritmo con la capacidad de analizar un tramo de video en tiempo real del tránsito vehicular, el modelo final fue capaz de diferenciar, clasificar y contar 3 tipos de vehículos utilizando IA y visión por computadora, con TensorFlow y OpenCV, los resultados determinaron un porcentaje superior al 90\% de efectividad si se contrastan con metodologías manuales o tradicionales.

Monterrey y Sosa \cite{Monterrey2020} implementaron un sistema de semaforización inteligente, eligiendo Amazon Rekognition para el procesamiento de imágenes debido a su alta fiabilidad, detectando el 100\% de los vehículos en las pruebas. Comparativamente, IBM Watson logró un 54.54\% y Google Cloud y Microsoft un 36.36\%. El sistema embebido se basó en una Raspberry Pi 3 modelo B, mientras que la cámara ELP 1MP USB se eligió con una resolución de 720p. Además, incorporaron un sensor de luz TSL2561 y un sensor de humedad DHT22 para monitorear las condiciones climáticas en las zonas de estudio, mejorando el rendimiento del sistema.

Álvarez y Olaya \cite{Alvarez2021} implementaron la red neuronal YOLOv3 para la detección en tiempo real de vehículos y peatones, entrenándola con un dataset de 2,100 imágenes extraídas de 20 videos simulados en un entorno de Unity 3D. El entrenamiento se realizó durante 100 épocas en PyTorch, optimizado en Google Colaboratory con una GPU Nvidia Tesla K80, logrando los niveles de precisión requeridos. Otro aporte es de Wong \cite{Eduardo2024} en la aplicación de esta red neuronal para la reducción del congestionamiento, determinando que es mucho más eficiente en comparación con otros modelos.

Jacobo \cite{Jacobo2015} utilizó Syncho 7 y Sim Traffic, junto con técnicas de detección de bordes y CNNs, para evaluar el tráfico. Obtuvo un ahorro anual de 181,800 litros de combustible y una reducción del 37\% en emisiones de CO. Otro aporte en la investigación de redes neuronales es de Mateus \cite{Mateus2014}, el cual presente la creación de Entornos Virtuales Inteligentes (EVI) con Redes Neuronales Artificiales (RNA) para el análisis de comportamientos dinámicos que se podría abordar en el congestionamiento de tránsito. Guzmán \cite{Felipe2022} planteó la utilización CNNs, a través de imágenes s tiempo real obtenidas de las cámaras en los semáforos, se toma una decisión para modificar el tiempo y priorizar aquellas vías que tengan un mayor congestionamiento, logrando un ahorro del 15.97\% a todos los carros presentes en el sistema.

Srinivasan\cite{Srinivasan2006} propuso un sistema multiagente distribuido para el control de señales de tráfico en tiempo real, utilizando redes neuronales híbridas y el teorema SPSA, logró optimizar las condiciones de tráfico con una reducción en el retraso promedio por vehículo en un 78\% y el tiempo de parada en un 85\%, en comparación con algoritmos tradicionales. \cite{Michel2013} presenta un controlador de semáforos basado en redes neuronales llamado EOMANN (Controlador basado en el Método de Observación del Entorno con Redes Neuronales Artificiales). El sistema diseñado fue evaluado en una intersección simulada utilizando el software SUM, logrando una reducción de más de 28.8 segundos con EOM-ANN. Mihai \cite{Lungu2024} estudia el uso de redes neuronales y algoritmos genéticos para mejorar la gestión del tráfico urbano al predecir y optimizar semáforos y rutas. 

 En los trabajos de Hernán \citeonline{Dario2018}, Peralta \cite{Peralta2017}, Bances \cite{Bances2014} y \cite{Jose2016}, diseñaron sistemas inteligentes para semáforos basados en hardware libre como Arduino, Raspberry Pi B+, PIC 18F2550 e incluso uso de sensores tales como el sensor Loop, detectores de  bucle para la recolección de datos en tiempo real, mediante la manipulación de los tiempos de encendido en las luces de los semáforos, ademas de recursos como OpenCV, Python e Highgui, logrando aumentar el flujo vehicular hasta 5.5\% y, disminuir el tiempo máximo de espera del vehículo para avanzar hasta un 18\%.

El trabajo de \cite{Arquimedes2019} plantea un modelo de implantación del sistema electrónico de semaforización inteligente, utilizando la técnica de conteo de vehículos en horas punta mediante el uso de sensores infrarrojos. Esta información es enviada a un sistema informático con un algoritmo de evaluación, de tal manera, que toma las decisiones de acuerdo al flujo vehicular que está establecido por nivel de congestión, registrando un 10\% en disminución de situaciones de congestión.

Montiel \cite{Sergio2021} empleó un vehículo flotante de monitoreo de tránsito para determinar el flujo vehicular haciendo uso de un modelo de lógica difusa y GPS, como resultado obtuvo mejoras de 54\% en la velocidad global y 36\% en tiempo de recorrido.

En consecuencia, los trabajos revisados destacan el uso de tecnologías avanzadas que proporcionan una base sólida para el desarrollo de un sistema de semaforización inteligente, aprovechando las tendencias tecnológicas para mejorar la movilidad urbana y contribuir al transporte en las ciudades.

\section{Prepare Your Paper Before Styling}
Before you begin to format your paper, first write and save the content as a 
separate text file. Complete all content and organizational editing before 
formatting. Please note sections \ref{AA}--\ref{SCM} below for more information on 
proofreading, spelling and grammar.

Keep your text and graphic files separate until after the text has been 
formatted and styled. Do not number text heads---{\LaTeX} will do that 
for you.

\subsection{Abbreviations and Acronyms}\label{AA}
Define abbreviations and acronyms the first time they are used in the text, 
even after they have been defined in the abstract. Abbreviations such as 
IEEE, SI, MKS, CGS, ac, dc, and rms do not have to be defined. Do not use 
abbreviations in the title or heads unless they are unavoidable.

\subsection{Units}
\begin{itemize}
\item Use either SI (MKS) or CGS as primary units. (SI units are encouraged.) English units may be used as secondary units (in parentheses). An exception would be the use of English units as identifiers in trade, such as ``3.5-inch disk drive''.
\item Avoid combining SI and CGS units, such as current in amperes and magnetic field in oersteds. This often leads to confusion because equations do not balance dimensionally. If you must use mixed units, clearly state the units for each quantity that you use in an equation.
\item Do not mix complete spellings and abbreviations of units: ``Wb/m\textsuperscript{2}'' or ``webers per square meter'', not ``webers/m\textsuperscript{2}''. Spell out units when they appear in text: ``. . . a few henries'', not ``. . . a few H''.
\item Use a zero before decimal points: ``0.25'', not ``.25''. Use ``cm\textsuperscript{3}'', not ``cc''.)
\end{itemize}

\subsection{Equations}
Number equations consecutively. To make your 
equations more compact, you may use the solidus (~/~), the exp function, or 
appropriate exponents. Italicize Roman symbols for quantities and variables, 
but not Greek symbols. Use a long dash rather than a hyphen for a minus 
sign. Punctuate equations with commas or periods when they are part of a 
sentence, as in:
\begin{equation}
a+b=\gamma\label{eq}
\end{equation}

Be sure that the 
symbols in your equation have been defined before or immediately following 
the equation. Use ``\eqref{eq}'', not ``Eq.~\eqref{eq}'' or ``equation \eqref{eq}'', except at 
the beginning of a sentence: ``Equation \eqref{eq} is . . .''

\subsection{\LaTeX-Specific Advice}

Please use ``soft'' (e.g., \verb|\eqref{Eq}|) cross references instead
of ``hard'' references (e.g., \verb|(1)|). That will make it possible
to combine sections, add equations, or change the order of figures or
citations without having to go through the file line by line.

Please don't use the \verb|{eqnarray}| equation environment. Use
\verb|{align}| or \verb|{IEEEeqnarray}| instead. The \verb|{eqnarray}|
environment leaves unsightly spaces around relation symbols.

Please note that the \verb|{subequations}| environment in {\LaTeX}
will increment the main equation counter even when there are no
equation numbers displayed. If you forget that, you might write an
article in which the equation numbers skip from (17) to (20), causing
the copy editors to wonder if you've discovered a new method of
counting.

{\BibTeX} does not work by magic. It doesn't get the bibliographic
data from thin air but from .bib files. If you use {\BibTeX} to produce a
bibliography you must send the .bib files. 

{\LaTeX} can't read your mind. If you assign the same label to a
subsubsection and a table, you might find that Table I has been cross
referenced as Table IV-B3. 

{\LaTeX} does not have precognitive abilities. If you put a
\verb|\label| command before the command that updates the counter it's
supposed to be using, the label will pick up the last counter to be
cross referenced instead. In particular, a \verb|\label| command
should not go before the caption of a figure or a table.

Do not use \verb|\nonumber| inside the \verb|{array}| environment. It
will not stop equation numbers inside \verb|{array}| (there won't be
any anyway) and it might stop a wanted equation number in the
surrounding equation.

\subsection{Some Common Mistakes}\label{SCM}
\begin{itemize}
\item The word ``data'' is plural, not singular.
\item The subscript for the permeability of vacuum $\mu_{0}$, and other common scientific constants, is zero with subscript formatting, not a lowercase letter ``o''.
\item In American English, commas, semicolons, periods, question and exclamation marks are located within quotation marks only when a complete thought or name is cited, such as a title or full quotation. When quotation marks are used, instead of a bold or italic typeface, to highlight a word or phrase, punctuation should appear outside of the quotation marks. A parenthetical phrase or statement at the end of a sentence is punctuated outside of the closing parenthesis (like this). (A parenthetical sentence is punctuated within the parentheses.)
\item A graph within a graph is an ``inset'', not an ``insert''. The word alternatively is preferred to the word ``alternately'' (unless you really mean something that alternates).
\item Do not use the word ``essentially'' to mean ``approximately'' or ``effectively''.
\item In your paper title, if the words ``that uses'' can accurately replace the word ``using'', capitalize the ``u''; if not, keep using lower-cased.
\item Be aware of the different meanings of the homophones ``affect'' and ``effect'', ``complement'' and ``compliment'', ``discreet'' and ``discrete'', ``principal'' and ``principle''.
\item Do not confuse ``imply'' and ``infer''.
\item The prefix ``non'' is not a word; it should be joined to the word it modifies, usually without a hyphen.
\item There is no period after the ``et'' in the Latin abbreviation ``et al.''.
\item The abbreviation ``i.e.'' means ``that is'', and the abbreviation ``e.g.'' means ``for example''.
\end{itemize}
An excellent style manual for science

\subsection{Authors and Affiliations}
\textbf{The class file is designed for, but not limited to, six authors.} A 
minimum of one author is required for all conference articles. Author names 
should be listed starting from left to right and then moving down to the 
next line. This is the author sequence that will be used in future citations 
and by indexing services. Names should not be listed in columns nor group by 
affiliation. Please keep your affiliations as succinct as possible (for 
example, do not differentiate among departments of the same organization).

\subsection{Identify the Headings}
Headings, or heads, are organizational devices that guide the reader through 
your paper. There are two types: component heads and text heads.

Component heads identify the different components of your paper and are not 
topically subordinate to each other. Examples include Acknowledgments and 
References and, for these, the correct style to use is ``Heading 5''. Use 
``figure caption'' for your Figure captions, and ``table head'' for your 
table title. Run-in heads, such as ``Abstract'', will require you to apply a 
style (in this case, italic) in addition to the style provided by the drop 
down menu to differentiate the head from the text.

Text heads organize the topics on a relational, hierarchical basis. For 
example, the paper title is the primary text head because all subsequent 
material relates and elaborates on this one topic. If there are two or more 
sub-topics, the next level head (uppercase Roman numerals) should be used 
and, conversely, if there are not at least two sub-topics, then no subheads 
should be introduced.

\subsection{Figures and Tables}
\paragraph{Positioning Figures and Tables} Place figures and tables at the top and 
bottom of columns. Avoid placing them in the middle of columns. Large 
figures and tables may span across both columns. Figure captions should be 
below the figures; table heads should appear above the tables. Insert 
figures and tables after they are cited in the text. Use the abbreviation 
``Fig.~\ref{fig}'', even at the beginning of a sentence.

\begin{table}[htbp]
\caption{Table Type Styles}
\begin{center}
\begin{tabular}{|c|c|c|c|}
\hline
\textbf{Table}&\multicolumn{3}{|c|}{\textbf{Table Column Head}} \\
\cline{2-4} 
\textbf{Head} & \textbf{\textit{Table column subhead}}& \textbf{\textit{Subhead}}& \textbf{\textit{Subhead}} \\
\hline
copy& More table copy$^{\mathrm{a}}$& &  \\
\hline
\multicolumn{4}{l}{$^{\mathrm{a}}$Sample of a Table footnote.}
\end{tabular}
\label{tab1}
\end{center}
\end{table}

\begin{figure}[htbp]
{\includegraphics{fig1.png}}
\caption{Example of a figure caption.}
\label{fig}
\end{figure}

Figure Labels: Use 8 point Times New Roman for Figure labels. Use words 
rather than symbols or abbreviations when writing Figure axis labels to 
avoid confusing the reader. As an example, write the quantity 
``Magnetization'', or ``Magnetization, M'', not just ``M''. If including 
units in the label, present them within parentheses. Do not label axes only 
with units. In the example, write ``Magnetization (A/m)'' or ``Magnetization 
\{A[m(1)]\}'', not just ``A/m''. Do not label axes with a ratio of 
quantities and units. For example, write ``Temperature (K)'', not 
``Temperature/K''.

\section*{Acknowledgment}

The preferred spelling of the word ``acknowledgment'' in America is without 
an ``e'' after the ``g''. Avoid the stilted expression ``one of us (R. B. 
G.) thanks $\ldots$''. Instead, try ``R. B. G. thanks$\ldots$''. Put sponsor 
acknowledgments in the unnumbered footnote on the first page.

\section{References}

\bibliographystyle{ieeetr}
\bibliography{referencias.bib}

\vspace{12pt}
\color{red}
IEEE conference templates contain guidance text for composing and formatting conference papers. Please ensure that all template text is removed from your conference paper prior to submission to the conference. Failure to remove the template text from your paper may result in your paper not being published.

\end{document}
